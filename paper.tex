% This template has been tested with LLNCS DOCUMENT CLASS -- version 2.20 (10-Mar-2018)

% !TeX spellcheck = en-US
% !TeX encoding = utf8
% !TeX program = pdflatex
% !BIB program = bibtex
% -*- coding:utf-8 mod:LaTeX -*-

% "a4paper" enables:
%  - easy print out on DIN A4 paper size
%
% One can configure a4 vs. letter in the LaTeX installation. So it is configuration dependend, what the paper size will be.
% This option  present, because the current word template offered by Springer is DIN A4.
% We accept that DIN A4 cause WTFs at persons not used to A4 in USA.

% "runningheads" enables:
%  - page number on page 2 onwards
%  - title/authors on even/odd pages
% This is good for other readers to enable proper archiving among other papers and pointing to
% content. Even if the title page states the title, when printed and stored in a folder, when
% blindly opening the folder, one could hit not the title page, but an arbitrary page. Therefore,
% it is good to have title printed on the pages, too.
%
% It is enabled by default as the springer template as of 2018/03/10 uses this as default

% German documents: pass ngerman as class option
% \documentclass[ngerman,runningheads,a4paper]{llncs}[2018/03/10]
% English documents: pass english as class option
\documentclass[english,runningheads,a4paper]{llncs}[2018/03/10]

%% If you need packages for other papers,
%% START COPYING HERE

% Set English as language and allow to write hyphenated"=words
%
% In case you write German, switch the parameters, so that the command becomes
\usepackage[english,main=ngerman]{babel}
%
% Even though `american`, `english` and `USenglish` are synonyms for babel package (according to https://tex.stackexchange.com/questions/12775/babel-english-american-usenglish), the llncs document class is prepared to avoid the overriding of certain names (such as "Abstract." -> "Abstract" or "Fig." -> "Figure") when using `english`, but not when using the other 2.
% english has to go last to set it as default language
%\usepackage[ngerman,main=english]{babel}
%
% Hint by http://tex.stackexchange.com/a/321066/9075 -> enable "= as dashes
\addto\extrasenglish{\languageshorthands{ngerman}\useshorthands{"}}
%
% Fix by https://tex.stackexchange.com/a/441701/9075
\usepackage{regexpatch}
\makeatletter
\edef\switcht@albion{%
  \relax\unexpanded\expandafter{\switcht@albion}%
}
\xpatchcmd*{\switcht@albion}{ \def}{\def}{}{}
\xpatchcmd{\switcht@albion}{\relax}{}{}{}
\edef\switcht@deutsch{%
  \relax\unexpanded\expandafter{\switcht@deutsch}%
}
\xpatchcmd*{\switcht@deutsch}{ \def}{\def}{}{}
\xpatchcmd{\switcht@deutsch}{\relax}{}{}{}
\edef\switcht@francais{%
  \relax\unexpanded\expandafter{\switcht@francais}%
}
\xpatchcmd*{\switcht@francais}{ \def}{\def}{}{}
\xpatchcmd{\switcht@francais}{\relax}{}{}{}
\makeatother

\usepackage{ifluatex}
\ifluatex
  \usepackage{fontspec}
  \usepackage[english]{selnolig}
\fi

\iftrue % use default-font
  \ifluatex
    % use the better (sharper, ...) Latin Modern variant of Computer Modern
    \setmainfont{Latin Modern Roman}
    \setsansfont{Latin Modern Sans}
    \setmonofont{Latin Modern Mono} % "variable=false"
    %\setmonofont{Latin Modern Mono Prop} % "variable=true"
  \else
    % better font, similar to the default springer font
    % cfr-lm is preferred over lmodern. Reasoning at http://tex.stackexchange.com/a/247543/9075
    \usepackage[%
      rm={oldstyle=false,proportional=true},%
      sf={oldstyle=false,proportional=true},%
      tt={oldstyle=false,proportional=true,variable=false},%
      qt=false%
    ]{cfr-lm}
  \fi
\else
  % In case more space is needed, it is accepted to use Times New Roman
  \ifluatex
    \setmainfont{TeX Gyre Termes}
    \setsansfont[Scale=.9]{TeX Gyre Heros}
    % newtxtt looks good with times, but no equivalent for lualatex found,
    % therefore tried to replace with inconsolata.
    % However, inconsolata does not look good in the context of LNCS ...
    %\setmonofont[StylisticSet={1,3},Scale=.9]{inconsolata}
    % ... thus, we use the good old Latin Modern Mono font for source code.
    \setmonofont{Latin Modern Mono} % "variable=false"
    %\setmonofont{Latin Modern Mono Prop} % "variable=true"
  \else
    % overwrite cmodern with the Times variant
    \usepackage{newtxtext}
    \usepackage{newtxmath}
    \usepackage[zerostyle=b,scaled=.9]{newtxtt}
  \fi
\fi

\ifluatex
\else
  % fontenc and inputenc are not required when using lualatex
  \usepackage[T1]{fontenc}
  \usepackage[utf8]{inputenc} %support umlauts in the input
\fi

\usepackage{graphicx}

% backticks (`) are rendered as such in verbatim environment. See https://tex.stackexchange.com/a/341057/9075 for details.
\usepackage{upquote}

% Nicer tables (\toprule, \midrule, \bottomrule - see example)
\usepackage{booktabs}

%extended enumerate, such as \begin{compactenum}
\usepackage{paralist}

%put figures inside a text
%\usepackage{picins}
%use
%\piccaptioninside
%\piccaption{...}
%\parpic[r]{\includegraphics ...}
%Text...

% For easy quotations: \enquote{text}
% This package is very smart when nesting is applied, otherwise textcmds (see below) provides a shorter command
\usepackage{csquotes}

% For even easier quotations: \qq{text}
\usepackage{textcmds}

%enable margin kerning
\RequirePackage[%
  babel,%
  final,%
  expansion=alltext,%
  protrusion=alltext-nott]{microtype}%
% \texttt{test -- test} keeps the "--" as "--" (and does not convert it to an en dash)
\DisableLigatures{encoding = T1, family = tt* }

%tweak \url{...}
\usepackage{url}
%\urlstyle{same}
%improve wrapping of URLs - hint by http://tex.stackexchange.com/a/10419/9075
\makeatletter
\g@addto@macro{\UrlBreaks}{\UrlOrds}
\makeatother
%nicer // - solution by http://tex.stackexchange.com/a/98470/9075
%DO NOT ACTIVATE -> prevents line breaks
%\makeatletter
%\def\Url@twoslashes{\mathchar`\/\@ifnextchar/{\kern-.2em}{}}
%\g@addto@macro\UrlSpecials{\do\/{\Url@twoslashes}}
%\makeatother

% Diagonal lines in a table - http://tex.stackexchange.com/questions/17745/diagonal-lines-in-table-cell
% Slashbox is not available in texlive (due to licensing) and also gives bad results. This, we use diagbox
%\usepackage{diagbox}

% Required for package pdfcomment later
\usepackage{xcolor}

% For listings
\usepackage{listings}
\lstset{%
  basicstyle=\ttfamily,%
  columns=fixed,%
  basewidth=.5em,%
  xleftmargin=0.5cm,%
  captionpos=b}%
\renewcommand{\lstlistingname}{List.}
% Fix counter as described at https://tex.stackexchange.com/a/28334/9075
\usepackage{chngcntr}
\AtBeginDocument{\counterwithout{lstlisting}{section}}

% Enable nice comments
\usepackage{pdfcomment}
%
\newcommand{\commentontext}[2]{\colorbox{yellow!60}{#1}\pdfcomment[color={0.234 0.867 0.211},hoffset=-6pt,voffset=10pt,opacity=0.5]{#2}}
\newcommand{\commentatside}[1]{\pdfcomment[color={0.045 0.278 0.643},icon=Note]{#1}}
%
% Compatibality with packages todo, easy-todo, todonotes
\newcommand{\todo}[1]{\commentatside{#1}}
% Compatiblity with package fixmetodonotes
\newcommand{\TODO}[1]{\commentatside{#1}}

% Bibliopgraphy enhancements
%  - enable \cite[prenote][]{ref}
%  - enable \cite{ref1,ref2}
% Alternative: \usepackage{cite}, which enables \cite{ref1, ref2} only (otherwise: Error message: "White space in argument")

% Doc: http://texdoc.net/natbib
\usepackage[%
  square,        % for square brackets
  comma,         % use commas as separators
  numbers,       % for numerical citations;
%  sort,          % orders multiple citations into the sequence in which they appear in the list of references;
  sort&compress, % as sort but in addition multiple numerical citations
                 % are compressed if possible (as 3-6, 15);
]{natbib}
% In the bibliography, references have to be formatted as 1., 2., ... not [1], [2], ...
\renewcommand{\bibnumfmt}[1]{#1.}

\ifluatex
  % does not work when using luatex
  % see: https://tex.stackexchange.com/q/419288/9075
\else
  % Prepare more space-saving rendering of the bibliography
  % Source: https://tex.stackexchange.com/a/280936/9075
  \SetExpansion
  [ context = sloppy,
    stretch = 30,
    shrink = 60,
    step = 5 ]
  { encoding = {OT1,T1,TS1} }
  { }
\fi

% Put footnotes below floats
% Source: https://tex.stackexchange.com/a/32993/9075
\usepackage{stfloats}
\fnbelowfloat

% Enable that parameters of \cref{}, \ref{}, \cite{}, ... are linked so that a reader can click on the number an jump to the target in the document
\usepackage{hyperref}
% Enable hyperref without colors and without bookmarks
\hypersetup{hidelinks,
  colorlinks=true,
  allcolors=black,
  pdfstartview=Fit,
  breaklinks=true}
%
% Enable correct jumping to figures when referencing
\usepackage[all]{hypcap}

\usepackage[group-four-digits,per-mode=fraction]{siunitx}

%enable \cref{...} and \Cref{...} instead of \ref: Type of reference included in the link
\usepackage[capitalise,nameinlink]{cleveref}
%Nice formats for \cref
\usepackage{iflang}
\IfLanguageName{ngerman}{
  \crefname{table}{Tab.}{Tab.}
  \Crefname{table}{Tabelle}{Tabellen}
  \crefname{figure}{\figurename}{\figurename}
  \Crefname{figure}{Abbildungen}{Abbildungen}
  \crefname{equation}{Gleichung}{Gleichungen}
  \Crefname{equation}{Gleichung}{Gleichungen}
  \crefname{listing}{\lstlistingname}{\lstlistingname}
  \Crefname{listing}{Listing}{Listings}
  \crefname{section}{Abschnitt}{Abschnitte}
  \Crefname{section}{Abschnitt}{Abschnitte}
  \crefname{paragraph}{Abschnitt}{Abschnitte}
  \Crefname{paragraph}{Abschnitt}{Abschnitte}
  \crefname{subparagraph}{Abschnitt}{Abschnitte}
  \Crefname{subparagraph}{Abschnitt}{Abschnitte}
}{
  \crefname{section}{Sect.}{Sect.}
  \Crefname{section}{Section}{Sections}
  \crefname{listing}{\lstlistingname}{\lstlistingname}
  \Crefname{listing}{Listing}{Listings}
}


%Intermediate solution for hyperlinked refs. See https://tex.stackexchange.com/q/132420/9075 for more information.
\newcommand{\Vlabel}[1]{\label[line]{#1}\hypertarget{#1}{}}
\newcommand{\lref}[1]{\hyperlink{#1}{\FancyVerbLineautorefname~\ref*{#1}}}

\usepackage{xspace}
%\newcommand{\eg}{e.\,g.\xspace}
%\newcommand{\ie}{i.\,e.\xspace}
\newcommand{\eg}{e.\,g.,\ }
\newcommand{\ie}{i.\,e.,\ }

%introduce \powerset - hint by http://matheplanet.com/matheplanet/nuke/html/viewtopic.php?topic=136492&post_id=997377
\DeclareFontFamily{U}{MnSymbolC}{}
\DeclareSymbolFont{MnSyC}{U}{MnSymbolC}{m}{n}
\DeclareFontShape{U}{MnSymbolC}{m}{n}{
  <-6>    MnSymbolC5
  <6-7>   MnSymbolC6
  <7-8>   MnSymbolC7
  <8-9>   MnSymbolC8
  <9-10>  MnSymbolC9
  <10-12> MnSymbolC10
  <12->   MnSymbolC12%
}{}
\DeclareMathSymbol{\powerset}{\mathord}{MnSyC}{180}

\ifluatex
\else
  % Enable copy and paste - also of numbers
  % This has to be done instead of \usepackage{cmap}, because it does not work together with cfr-lm.
  % See: https://tex.stackexchange.com/a/430599/9075
  \input glyphtounicode
  \pdfgentounicode=1
\fi

% correct bad hyphenation here
\hyphenation{op-tical net-works semi-conduc-tor}

%% END COPYING HERE


% Add copyright
% Do that for the final version or if you send it to colleagues
\iffalse
  %state: intended|submitted|llncs
  %you can add "crop" if the paper should be cropped to the format Springer is publishing
  \usepackage[intended]{llncsconf}

  \conference{name of the conference}

  %in case of "llncs" (final version!)
  %example: llncs{Anonymous et al. (eds). \emph{Proceedings of the International Conference on \LaTeX-Hacks}, LNCS~42. Some Publisher, 2016.}{0042}
  \llncs{book editors and title}{0042} %% 0042 is the start page
\fi

% For demonstration purposes only
\usepackage[math]{blindtext}
\usepackage{mwe}


\begin{document}

\title{Paper Title}
%If Title is too long, use \titlerunning
%\titlerunning{Short Title}

%Single insitute
\author{Firstname Lastname \and Firstname Lastname}
%If there are too many authors, use \authorrunning
%\authorrunning{First Author et al.}
\institute{Institute}

%% Multiple insitutes - ALTERNATIVE to the above
% \author{%
%     Firstname Lastname\inst{1} \and
%     Firstname Lastname\inst{2}
% }
%
%If there are too many authors, use \authorrunning
%  \authorrunning{First Author et al.}
%
%  \institute{
%      Insitute 1\\
%      \email{...}\and
%      Insitute 2\\
%      \email{...}
%}

\maketitle

\begin{abstract}
  \lipsum[1]
\end{abstract}

\begin{keywords}
  keyword1, keyword2
\end{keywords}

\section{März}
\subsection{1. Woche}
\subsubsection{2.03}
- pc aufsetzen
- dual boot win ubuntu
- meeting db technik
- db monitoring checks

\subsubsection{3.03}
- Cisco ASA 5506-X Einstellungen rumspielen über Kommandozeile und dann über java Oberfläche
- Hardware am pc anschließen
- paul brandt stichwörter IT-Sicherheit
- Richard und Stefan über Schulter gucken bei Firewall Einstellungne ASA mit Kundenanforderungen

\subsubsection{4.03}
-Information Security Managment System ISO Zertifizierung mit Mathias, Kernbetrieb, vom Kunden gewünscht, Dokumentation, Ticketsystem
-- Angriffszenarien 
---DDOS Attacke über ASA erkannt, Eintragung in IP Tables über Linux Firewall bzw. IP bannen
--- Kontaktformular versenden von Spam Mails führt zu Ban durch Provider(web.de gmx strato)
--- Sicherheitsschwachstelle in Webserver CMS, welches vom Kundne verwaltet wird, führt auch zu Spammailversand
- Ticket von Vorfall, Erkennung und dann IT-Forensik bei Bitcoin Trojaner
- Webserver in virtual box zum Rumspielen

\subsubsection{5.03}
- Webserver Pentesting, Cross Site scripting xss
- Richard über Schulter gucken bei Einrichtung von ASA und Firewall Regeln

\subsubsection{6.03}
- Sicherer Umgang mit Datenentfernung
- rumliegende sd Karte vorm Firmengebäude 
- isoliertes Gerät, alter, rumliegender laptop, kein wlan zugangsdaten, Festplatte ausgebaut, Knoppix live usb starten
- mit gparted partition von sd /dev/mmcblk0 gelöscht
- shred -vzn 0 /dev/mmcblk0 mit Nullen überschreiben
- usb stick mit knoppix plattmachen sudo umount -l /dev/sda1
- mit fdisk -l usb drive erkennen
- shred -vzn 0 /dev/sda
- im Rechenzentrum switch anschließen und port einrichten
- metasploit aufsetzen über vm
- sitzung gespräch security vorfall sd karte, corona topdown unternehmensführung, selbstbestimmte, eigenverantwortliche mitarbeiter, Firmenfeier
\subsection{2. Woche}
\subsubsection{9.03}
- Dienstbesprechung
- metasploitable3 internalnetworking in virtualbox konfigurieren, kein Nat um nicht Firmennetzwerk zu gefährden
- kali linux im internal network

\subsubsection{10.03}
- wordpress metasploit mit kali linux durchgeführt
- port 8585
- TARGETURI	 /wordpress/
- FORM PATH	/index.php/king-of-hearts/ 
- exploit/multi/http/wpninjaformsunauthenticatedfileupload

-- Apache Struts port 8282 gets working meterpreter shell $exploit/multi/http/struts_dmi_rest_exec$

--- Glassfish 4848, 8080, ssl true, target 1 java universal gets working meterpreter shell
$multi/http/glassfish_deployer$

\subsubsection{11.03}
- difference between remote and local perspective port scanning
- access network through compromised machines
- port forwarding with Meterpreter
--forwarded connections from a local port 3306 kali linux, over Meterpreter to a local port 3306 on win2k8
--allowed access port 3306 on Metasploitable3 from a remote
$https://www.hackingtutorials.org/metasploit-tutorials/metasploitable-3-port-forwarding/$
- mysqld uses port 3306 can be accessed as local
--maria db sees wordpress database, get admin password through john and change wordpress website set king of diamonds from private to public

\subsubsection{12.03}
- Richard und Stefan über die Schulter geguckt beim VPN einrichten mittels ASA Java Oberfäche
- Wordpress Blog eingerichtet, Einrichtung von apache, mysql, php

\subsubsection{13.03}
- zweiwöchige Besprechung der Praktikumsaufgaben, MACSec für planet area aufbauen
- metasploit, mysql Datenbankenaufbau nachvollziehen, zugriff über portforwarding, verschiedene Dateiformate wiederherstellen
- wordpress zugriff auf wp-content/uploads

\subsection{3. Woche}
\subsubsection{16.03}
- Macsec Verbindung zwischen zwei Switches 
 $https://www.cisco.com/c/en/us/td/docs/switches/lan/catalyst9300/software/release/16-6/configuration_guide/sec/b_166_sec_9300_cg/macsec_encryption.html$
-mka Policy
\begin{lstlisting}[language=bash]
Switch(config)# mka policy mka_policy 
Switch(config-mka-policy)# macsec-cipher-suite gcm-aes-256 
Switch(config-mka-policy)# confidentiality-offset 30 
Switch(config-mka-policy)# end 
- keys
Switch(config)# Key chain keychain1 macsec 
Switch(config-key-chain)# key 1000 
Switch(config-keychain-key)# cryptographic-algorithm gcm-aes-256
Switch(config-keychain-key)# key-string 12345678901234567890123456789012 
Switch(config-keychain-key)# lifetime local 12:12:00 July 28 2016 12:19:00 July 28 2016 
Switch(config-keychain-key)# end 
-interface 
Switch(config)# interface GigabitEthernet 0/0/0 
Switch(config-if)# mka policy mka_policy 
Switch(config-if)# mka pre-shared-key key-chain keychain1
Switch(config-if)# end 

\end{lstlisting}
-Trustsec
\begin{lstlisting}[language=bash]
Switch# configure terminal
Switch(config)# interface tengigabitethernet 1/1/2
Switch(config-if)# cts manual
Switch(config-if-cts-manual)# sap pmk 1234abcdef mode-list gcm-encrypt null no-encap
Switch(config-if-cts-manual)# no propagate sgt
Switch(config-if-cts-manual)# exit
Switch(config-if)# end
\end{lstlisting}
-1gbit mit iperf 777mbits, 10gi kann mit laptops nicht getestet werden
- laptops ips im netzwerk zuweisen 192.168.111.1, 192.168.111.2 /24
\begin{lstlisting}[language=bash]
iperf -c 192.168.111.1
iperf -s 
\end{lstlisting}

\subsubsection{17.03}
- versucht trustsec authentication mit seed device
\begin{lstlisting}[language=bash]
 Device(config)# aaa new-model
Device(config)# aaa authentication dot1x default group radius
Device(config)# aaa authorization network MLIST group radius
Device(config)# cts authorization list MLIST
Device(config)# aaa accounting dot1x default start-stop group radius
Device(config)# radius-server host 10.20.3.1 auth-port 1812 acct-port 1813 pac key AbCe1234
Device(config)# radius-server vsa send authentication
Device(config)# dot1x system-auth-control
Device(config)# exit 
\end{lstlisting}
-für non seed device auf dem zweiten switch
\begin{lstlisting}[language=bash]
 Device(config)# aaa new-model
Device(config)# aaa authentication dot1x default group radius
Device(config)# aaa authorization network MLIST group radius
Device(config)# aaa accounting dot1x default start-stop group radius
Device(config)# radius-server vsa send authentication
Device(config)# dot1x system-auth-control
Device(config)# exit 
\end{lstlisting}
- wenn $cts manual$ das interface manuell konfiguriert dann gibt es keine autehtication, $cts dot1x (config-if-cts-dot1x) $gabs nicht, Veraltete Dokumentationsanleitung da es cli nicht mehr gibt \url{https://community.cisco.com/t5/switching/configuration-radius-c9300-48p/td-p/3730127}

-You can manually configure Cisco TrustSec on an interface. You must manually configure the interfaces on both ends of the connection. No authentication occurs; policies can be statically configured or dynamically downloaded from an authentication server by specifying the server’s device identity. \url{https://www.cisco.com/c/en/us/td/docs/switches/lan/trustsec/configuration/guide/trustsec/ident-conn_config.html#65496}

-anderer laptop 935mbtis bei iperf

- Tätigkeiten im Unternehmen/beim Kunden, Ansprechpartner fallen wegen Corona möglicherweise aus, andere Aufgaben vorziehen

-vpn zugang eingerichtet
Configdatei
- drei Management Zeilen löschen
- group nogroup ändern
Installationspackages
- sudo apt install openvpn-systemd-resolved/bionic
- sudo apt install network-manager-openvpn-gnome
- sudo apt install network-manager-openvpn
- selber apt install openvpn
dann abmelden/anmelden

\subsubsection{18.03}
-\url{https://www.cisco.com/c/dam/en/us/solutions/collateral/enterprise/design-zone-security/how_to_intro_macsec_ndac_guide.pdf}
-authentication Problem, für den seed device(swtest1 switch) müsste ein Trustsec Server angegeben werden
- Cisco Identity Services Engine (ISE) muss Cisco Software auf einem Policy Service Node installieren
- ist so ein Network Device Admission Control (NDAC) für andere geräte schon eingerichtet?
-\url{https://www.cisco.com/c/en/us/support/switches/catalyst-9300-24x-a-switch/model.html} --> configurations guides --> trustsec switch configuration guide

\subsubsection{19.03}
- Quality of Service (QoS)z.B Voice und Data
- "The 10-Gigabit interfaces do not support auto-QoS for VoIP with Cisco IP Phones or with devices running the Cisco SoftPhone feature." veraltet' 12-2-55 vs 16-12 im LInk
\url{https://www.cisco.com/c/en/us/td/docs/switches/lan/catalyst3750/software/release/12-2_55_se/configuration/guide/scg3750/swqos.html#91698}
-auto qos geht anscheinend doch
\url{https://www.cisco.com/c/en/us/td/docs/switches/lan/catalyst9300/software/release/16-12/configuration_guide/qos/b_1612_qos_9300_cg/configuring_auto_qos.html#reference_kcp_nz5_41b}
\begin{lstlisting}[language=bash]
-Device(config)# interface HundredGigE1/0/2
Device(config-if)# auto qos trust cos
Device(config-if)# end
Device# show policy-map interface HundredGigE1/0/2
\end{lstlisting}
- mit auto qos global compact werden nur die configuration  messages versteckt (hidden)
-confidentiality offset ermöglicht qos?
\subsubsection{20.03}
-lacp link Bündelung, proprietäre Cisco PAgP implementierung
\url{https://www.omnisecu.com/cisco-certified-network-associate-ccna/how-to-configure-etherchannel-port-aggregation-protocol-pagp-in-cisco-switch.php}
\begin{lstlisting}[language=bash]
omnisecu.com.SW1(config)#interface range gigabitEthernet 0/1 - 2
omnisecu.com.SW1(config-if-range)#channel-group 1 mode desirable
omnisecu.com.SW1(config-if-range)#channel-protocol pagp 
omnisecu.com.SW1(config-if-range)#exit
omnisecu.com.SW1(config)#exit
\end{lstlisting}
-tcpdump, wireshark um auf monitorport pakete abzuhören
-zweite netzwerkkarte an Laptop und switch angeschlossen
-scapy versucht um ein Paket loszuschicken, hat nicht geklappt,a ber auch nicht erforderlich weil Wireshark bei MacSec das als Protokoll anzeigen würde und nicht tcp
-port syn scanning
- wireshark tutorial angeguckt
- monitor port mit monitor session 1 source te1/1/8
.monitor session 1 destinatin te1/0/24 encasulation replicate, replicate bringt wohl nichts


\subsection{4. Woche}
\subsubsection{23.03}
-problem bei cisco trust sec bei switch 2 kommt nach " show macsec interface te1/1/8" immer$ "*Mar 23 09:30:01.940: MACSec-IPC: getting macsec sa_sc response",$ also lieber mka single host mode probieren
-no macsec network-link auf allen nodes, dann mka policy ändern, dann wieder macsec network link anmachen \url{https://www.cisco.com/c/en/us/td/docs/switches/lan/catalyst3650/software/release/16-6/configuration_guide/sec/b_166_sec_3650_cg/macsec_encryption.html}
-fehlersuche mit stefan taute, usb netzwerkkarte kann nur 100mbit/s deswegen DUMP größe zwischen eno1 und usbnetzwerkkarte 1GB zu 15MB
- bei 100mbits switch beschränkung 118 und 124 MB
-monitorport am switch sieht eventuell bereits die entschlüsselten daten 

\subsubsection{24.03}
-linphone eingerichtet
- Ziel ist Hub zwischenzuschalten, um die Pakete abzuhören
- Nur Switch vorhanden, deshalb macspoofing mittels macof, um die mac adressen tabelle überlaufen zu lassen, sodass switch broadcasted und linux box mit usb netzwerkkabel mithören kann \url{https://brakertech.com/flood-network-with-random-mac-addresses-with-macof-tool/}
-  sudo macof -i eno1
führt dazu, dass ping nach 192.168.111.1 durch sudo tcpdump -i enx74da389fee2f -s 0 -w hubping.dump zugesendet bekommt
- ohne macsec funktioniert dies, allerdings lässt sich macsec auf diesem interface te1/0/12 nicht anstellen, sondern nur auf te1/1/8 ( - Interface is not MACsec capable.
)

\subsubsection{25.03}
- MACSec Pakete werden erfolgreich auf zwischengeschaltetem Switch auf Ports 12 mitgeschnitten und auf Port 3 des kleinen Switches per USB Netzwerkkarte an Linux Box übertragen
-funktioniert entweder mit CTS(Trustsec) oder MKA 
-cisco 9300 24ux a Seite --> Configuration Guides --> Platform COnfiguration -->Software Configuration Guide, Cisco IOS XE Gibraltar 16.12.x (Catalyst 9300 Switches) --> Security --> MACSec Encryptino ist die beste offizielle Dokumentation \url{https://www.cisco.com/c/en/us/td/docs/switches/lan/catalyst9300/software/release/16-12/configuration_guide/sec/b_1612_sec_9300_cg/macsec_encryption.html}
- beste Configexample mit LACP
\begin{lstlisting}[language=bash]
Device> enable
Device# configure terminal
Device(config)# key chain KC macsec
Device(config-key-chain)# key 1000
Device(config-key-chain)# cryptographic-algorithm aes-128-cmac
Device(config-key-chain)# key-string FC8F5B10557C192F03F60198413D7D45
Device(config-key-chain)# exit
Device(config)# mka policy POLICY
Device(config-mka-policy)# key-server priority 0
Device(config-mka-policy)# macsec-cipher-suite gcm-aes-128
Device(config-mka-policy)# confidentiality-offset 0
Device(config-mka-policy)# exit
Device(config)# interface gigabitethernet 1/0/1
Device(config-if)# channel-group 2 mode active
Device(config-if)# macsec network-link
Device(config-if)# mka policy POLICY
Device(config-if)# mka pre-shared-key key-chain KC
Device(config-if)# exit
Device(config)# interface gigabitethernet 1/0/2
Device(config-if)# channel-group 2 mode active
Device(config-if)# macsec network-link
Device(config-if)# mka policy POLICY
Device(config-if)# mka pre-shared-key key-chain KC
Device(config-if)# end
\end{lstlisting}

\subsubsection{26.03}
- PAgP auf Kupferkabeln eingerichtet, geht mit macsec
- Switch im TGZ keller eingebaut
- 2. Switch im 1. OG Haus 3 bei Logic Way hingestellt, Schrauben passten nicht, PatchKabel hatte auch noch kein Signal
-Prüfung Notstromaggregat
\subsubsection{27.03}
-dkb grund de umzug in testlab überlegt, Nutzen?
- Testsezenarien für LACP Stecker rausziehen, ausfallsicherung, doppelte Datengeschwindikeit
- Angebot für Kunden, Redundanz, Geschwindigkeit bei Verschlüsselung
- MdK kann sich ipSec, Vpn sparen und LAN mit Macsec nutzen
\subsection{5. Woche}
\subsubsection{30.03}
-LACP Bündelung testen, Ausfallsicherheit, doppelte Datenrate \url{https://www.cisco.com/c/en/us/td/docs/switches/lan/catalyst9300/software/release/16-12/configuration_guide/sec/b_1612_sec_9300_cg/macsec_encryption.html}
-show etherchannel summary
- MACSec an/ausschalten bei Abschnitt Configuring MKA MACsec using PSK no macsec network-link
\begin{lstlisting}[language=bash]
Device> enable
Device# configure terminal
Device(config)# key chain KC macsec
Device(config-key-chain)# key 1000
Device(config-key-chain)# cryptographic-algorithm aes-128-cmac
Device(config-key-chain)# key-string FC8F5B10557C192F03F60198413D7D45
Device(config-key-chain)# exit
Device(config)# mka policy POLICY
Device(config-mka-policy)# key-server priority 0
Device(config-mka-policy)# macsec-cipher-suite gcm-aes-128
Device(config-mka-policy)# confidentiality-offset 0
Device(config-mka-policy)# exit
Device(config)# interface gigabitethernet 1/0/1
Device(config-if)# channel-group 2 mode active
Device(config-if)# macsec network-link
Device(config-if)# mka policy POLICY
Device(config-if)# mka pre-shared-key key-chain KC
Device(config-if)# exit
Device(config)# interface gigabitethernet 1/0/2
Device(config-if)# channel-group 2 mode active
Device(config-if)# macsec network-link
Device(config-if)# mka policy POLICY
Device(config-if)# mka pre-shared-key key-chain KC
Device(config-if)# end

Layer 2 EtherChannel Configuration

Device 1

Device> enable
Device# configure terminal
Device(config)# interface port-channel 2
Device(config-if)# switchport
Device(config-if)# switchport mode trunk
Device(config-if)# no shutdown
Device(config-if)# end

Device 2

Device> enable
Device# configure terminal
Device(config)# interface port-channel 2
Device(config-if)# switchport
Device(config-if)# switchport mode trunk
Device(config-if)# no shutdown
Device(config-if)# end
\end{lstlisting}
- ausfallsicherheit funktioniert
- datengeschwindigkeit bleibt für singel tcp stream gleich und verdoppelt sich nicht
- müsste mit weiterem src mac getestet werden, da load balancing standardmäßig darauf setzt
-Jit.si Server in kvm Container aufsetzen 192.168.183.158
- dns-nameservers 195.98.223.1
-proxmox einstellungen mit lukas
-ssh root@192.168.183.158 ,pw abbrechen um file zu erzeugen
- nano $/root/.ssh/authorized_keys$

\subsubsection{31.03}
- clean install nach quickanleitung führt zu Verbindung nicht verfügbar, müssten nginx logs angucken warum, aber noch nicht getan
- deinstallation und neuinstallation konferenz aufmachen möglich, aber verbindung bricht ab sobald zweiter teilnehmer joint
- jitsi videobridge component secret erforderlich, weiß aber keinen wert dafür
- jitsi videobridge (jvb) /etc/jitsi/videobridge/config, (prosody) xmpp server /etc/prosody/conf.avail/192.168.183.158.cfg.lua, (jicofo) jitsi conference focus logs und configs angeguckt, aber schwierige fehlersuche
\url{https://github.com/jitsi/jitsi-meet/blob/master/doc/quick-install.md}
-- versuch mit apache2 default website anzeigen ging nicht, da  firefox hat immer auf ssl umgeleitet, weil das vorher bei jitsi so konfiguriert war und die deinstalltion von jitsi nicht restlos möglich war
-- Lösung durch privates fenster ohne cache/cookie die seite aufrufen
- vorher funktionierenden nginx server installieren, dann mit normalem quick install beginnen (nicht extra ohne jitsi-meet-turnserver), da nginx nur auf port 80 und nicht 443 konfiguriert ist, aber hat eventuell, doch ohne coturn installiert

\subsubsection{1.04}
-Jitsi Server Test vorbereiten, Gesprächscounter geht nicht? jicofo can focus im log nicht als component laden?
- Testkriterien Browser Firefox oder Chrome, OS Linux oder Windows, Cpu Auslastung, Anzahl Teilnehmer mit Bildschirmübertragung Webcam/micro
- h.264 seitdem geht firefox nicht mehr, sondern nur noch chrome
- Jitsi Desktop xmpp (jabber pidgin) und sip eingerichtet, aber kann nicht mit jitsi meet verbinden, daher nutzlos
- LACP Load Sharing testen src mac adresse standard
- usb netzwerkkarten anschließen geht nicht wegen routing
- iperf tests mit mehrern servern und clients um doppelte bandbreite zu nutzen

\subsubsection{2.04}
- Calc JitsiTestKonferenz Tabelle vorbereitet
- Nextcloud bei talk.planet-ic.de damit online geshared die Werte eingetragen werden können-
- zwei weitere Laptops an lokales Netz angschlossen, um Konferenz zu testen, da vorher immer ab drittem teilnehmer von mir selbst ein abbruch der Konferenz erfolgte, lag wahrscheinlich an p2p zu server umstellung
- Problem war Mitarbeiterfirewall der testlab die alles blockt
- herausgefunden durch jvb.log pair failed zu ip von meinem laptop oder den windows laptops
- vorher namp scan auf testlab server 192.168.183.158, um zu sehen, dass ich nicht freigeschaltet war
- jetzt gehen firefox und chrome
- überlegen optimierungen , das beim browser für webrtc die codecs angeboten werden vp8 vp9 h264 \url{chrome://webrtc-internals/}
-lacp mit vier laptops(zwei server zwei clients) geht mit iperf3, erreicht dopplete bandbreite
- drei clients auf einen server mit drei ports geht nur 100mbits
- zweite nic usb netzwerkkarte mit zwei client und einem server gibt auch nur 50+50 = 100mbits

\subsubsection{3.04}
- überlegt mit focus@auth.192.168.183.158 componenent not connected warn, bouncing error
- speakerstats und conferenzdurationtimer sind wegen fehlendem zertifikat nicht auf port 5281 ferigeschaltet, angeblich aber kein problem, nochmal log von prosody und jicofo angucken
-Jitsi Testkonferenz durchgeführt und performancewerte gesammelt
- chrome skype gastkonto konferenz
- multicast gibts in iperf3 nicht mehr

\subsection{6. Woche}
\subsubsection{06.04}
- Aufbau LACP Geschwindigkeitstest 2 * 100 MBit/s und Standardwert bei Loadbalancing, also src-mac adresse
- 1 Laptop ist iperf3 Server mit drei weiteren USB 3.0 Netzwerkkarten, deren IP statisch festgelegt werden
- vier weitere Laptops sind die iperf3 Clients mit passender IP für eines der vier Netze
- alle in verschiedenen Netzen 
- voher alle am gleichen Switch angeschlossen, um 1Gbit (~940Mbit) zu bestätigen
- Danach Sender und Empfänger auf die gegenüberliegenden Seiten der LACP Brücke/Verbindung gesteckt
\begin{lstlisting}[language=bash]
Laptop 1 Server mit drei USB Netzwerkkarten
$ iperf3 -s (192.168.111.1/24)
$ iperf3 -s -p5202 (192.168.112.42/24)
$ iperf3 -s -p5203 (192.168.113.60/24)
$ iperf3 -s -p5204 (192.168.114.177/24)

Laptop 2 erstes Netz
$ iperf3 -c 192.168.111.1 -t600
Laptop 3 zweites Netz
$ iperf3 -c 192.168.112.42 -t600 -p5202
Laptop 4 drittes Netz
$ iperf3 -c 192.168.113.60 -t600 -p5203
Laptop 5 viertes Netz
$ iperf3 -c 192.168.114.177 -t600 -p5204
\end{lstlisting}
- bei zwei Verbindungen erreicht macsec ~93Mbit/s jeweils
- drei Verbindungen verteilt sich zu 100 + 2*50 Mbit/s
- vier Verbindungen 2*50Mbit/s + 2*50Mbit/s
\begin{table}
	\caption{10min LACP iperf3 mit MACSec}
	\label{tab:LACPMACSec}
	\centering
	\begin{tabular}{lr}
		\toprule
		Port & Bandwidth in MBit/s\\
		\midrule
		5201      & 35.8      \\
		5202     & 51.6     \\
		5203      & 41.3      \\
		5204     & 56.6     \\
		\bottomrule
		5201 +5204     & 35.8 + 56.6 = 92.4      \\
		5202 +5203     & 51.6 + 41.3 = 92.9      \\
	\end{tabular}
\end{table}
- Verbindung 1 und 4 teilen sich; 35.8 + 56.6 = 92.4
- Verbindung 2 und 3 teilen sich; 51.6 + 41.3 = 92.9

\begin{table}
	\caption{10min LACP iperf3 ohne MACSec}
	\label{tab:LACPohneMACSec}
	\centering
	\begin{tabular}{lr}
		\toprule
		Port & Bandwidth in MBit/s\\
		\midrule
		5201      & 45.2      \\
		5202     & 50.5    \\
		5203      & 44.3      \\
		5204     & 49.3     \\
		\bottomrule
		5201 +5204     & 45.2 + 49.3 = 94.5      \\
		5202 +5203     & 50.5 + 44.3 = 94.8      \\
	\end{tabular}
\end{table}

- Windows file sharing über lan: netzwerkverbindungen --> erweiterte fregabeeinstallungen anstellen
- ordner freigeben und berechtigungen setzen 
- be anderem pc pfad \path{\\PC_SCH01\\iperf3-win1414\}
	
\subsubsection{07.04}
-Tabelle \ref{tab:KupferMACSec} stellt Messung 10min einfache Verbindung über Kupferkabel 1GBit/s mit MACSec dar
\begin{table}
	\caption{10min Kupferkabel iperf3 mit MACSec}
	\label{tab:KupferMACSec}
	\centering
	\begin{tabular}{llr}
		\toprule
		Interval& Transfer  & Bandwidth \\
		\midrule
		0.00-600.04 sec &65.2 GBytes      & 934 MBits/sec     \\
		\bottomrule
	\end{tabular}
\end{table}
-Tabelle \ref{tab:KupferohneMACSec} zeigt Messung 10min einfache Verbindung über Kupferkabel 1GBit/s ohne MACSec --> exakt gleiche Werte
\begin{table}
	\caption{10min Kupferkabel iperf3 ohne MACSec}
	\label{tab:KupferohneMACSec}
	\centering
	\begin{tabular}{llr}
		\toprule
		Interval& Transfer  & Bandwidth \\
		\midrule
		0.00-600.04 sec &65.2 GBytes      & 934 MBits/sec     \\
		\bottomrule
	\end{tabular}
\end{table}


-gns3 webui mit host only adapter ip port 3080
- gns3 gui, um router images zu importieren \url{https://docs.gns3.com/1QXVIihk7dsOL7Xr7Bmz4zRzTsJ02wklfImGuHwTlaA4/index.html}
- versucht router mit zweitem netzwerkadapter nat zum internet zu verbinden

\subsubsection{08.04}
- jitsi test laptops vorbereitet 2*i5 mit windows, 1*i7 windows 1*i7 ubuntu
-gns3 iosvl2 image für switch, lacp eingerichet, ausfallsicherheit klappt
- virtualbox images in die topologie eingefügt, win2k8 server metasploit und 2*kali linux, alles im selben netz über switch, können sich anpingen
-iperf aber nur 1.75mbits von kali(vm)-->switch<--kali(vmlinked base und headless)
- nur virtualbox kali zu host hatte 900mbits

\subsubsection{09.04}
- gns mit L2 IOU verliert immer die Verbindung zum Switch nach 30s iperf3, erreicht 40mbits
- drei Kali LInux VirtualbBox VMs und einmal win10 Virtualbox VM eingebunden
- nochmals LACP nachgestellt, kann aber nicht getestet werden wegen cisco licensen für gns3/verbindungsabbruch
- aufgabe backbone area von planet ic nachstellen mit dynamips c7200 router und BGP und OSPF

\subsection{7. Woche}
\subsubsection{14.04}
-vier Laptops mit firewall nach außen freigeschaltet
- in Tabelle \ref{tab:JitsiVergleich} haben alle Anbieter sehr ähnliche Werte
-i5 win 172.16.99.100 
 i5 Präsentation/Win 172.16.99.175 OHNE WEBCAM
 i7 win 172.16.99.176
 i7 linux 172.16.99.79

- \url{chrome://gpu} linux schlechter als windwos da kein hardware accelerated video decode -\url{https://www.omgubuntu.co.uk/2018/10/hardware-acceleration-chrome-linux} chrome hat auf linux kein interesse dies zu implementieren
 
 \begin{table}
 	\caption{Vergleich Gesamt CPU Auslastung in \% jitsi vier Teilnehmer}
 	\label{tab:JitsiVergleich}
 	\centering
 	\begin{tabular}{lccc}
 		\toprule
 		Laptop & 192.168.183.158& meet.jit.si  & copendia \\
 		\midrule
 		i7 linux &50 &55      & 53     \\
 		i7 win &31 &33      & 33     \\
 		i5 Präsentation/Win &31 &31      & 31     \\
 		i5 Win &50 &56     & 52     \\ 		
 		\bottomrule
 	\end{tabular}
 \end{table}

-ospf in gns3 mit backbone area nachstellen \url{https://docs.gns3.com/1d1huu6z9-wWGD_ipTSQZqy2mpaxiqzymu-YQo6at_Jg/index.html}

\subsubsection{15.04}
-gns3 backbone area mit drei c7200 routern und ospf und bgp nachgebaut, sodass sie ihre loopback adressen pingen können
-bgp configuring peer process und bgp routing process abschnitte auf allen routern durchführen und ips und Autonomous system(AS) anpassen \url{https://www.cisco.com/c/en/us/td/docs/ios-xml/ios/iproute_bgp/configuration/15-mt/irg-15-mt-book.pdf}
\begin{lstlisting}[language=bash]
BGPRoutingProcess
Device>enable
Device#configure terminal
Device(config)#router bgp 40000
Device(config-router)#network 2.2.2.2 mask 255.255.255.255
Device(config-router)#end
Device#show ip bgp
\end{lstlisting}
\begin{lstlisting}[language=bash]
BGPPeer
Device>enable
Device#configure terminal
Device(config)#router bgp 40000
Device(config-router)#neighbor 20.1.1.3 remote-as 45000
Device(config-router)#address-family ipv4 unicast
Device(config-router-af)#neighbor 20.1.1.3 activate
Device(config-router)#end
Device#show ip bgp
Device(config-router-af)#show ip bgp neighbors
\end{lstlisting}

-ospf \url{https://docs.gns3.com/1d1huu6z9-wWGD_ipTSQZqy2mpaxiqzymu-YQo6at_Jg/index.html}
-auf allen geräten interfaces konfigurieren
\begin{lstlisting}[language=bash]
Configure Layer 3 addressing on the devices
R1(config)# int fa0/0
R1(config-if)#no shut
R1(config-if)#ip add 10.1.1.1 255.255.255.0
R1(config-if)#int loop 0
R1(config-if)#ip add 1.1.1.1 255.255.255.255
R1(config-if)# end
\end{lstlisting}
-ospf einrichten auf allen geräten
\begin{lstlisting}[language=bash]
R1(config)#router ospf 1
R1(config-router)#router-id 1.1.1.1
R1(config-router)#network 0.0.0.0 255.255.255.255 area 0
R1(config-router)#end
note:  using "network 0.0.0.0 255.255.255.255 area 0" in ospf is a shortcut to enable ospf on all interfaces in an ospf area. It's not always desirable in the real world, but is fine for lab purposes
R2#sh ip ospf neigh
R1#sh ip route
\end{lstlisting}
-in tabelle \ref{tab:Jitsi1314} jitsi testkonferenz mit 13/14 teilnehmern und performancewerte der vier neu aufgestzten laptops aufgeschrieben
 \begin{table}
	\caption{Vergleich Gesamt CPU Auslastung in \% jitsi 13/14 Teilnehmer}
	\label{tab:Jitsi1314}
	\centering
	\begin{tabular}{lccc}
		\toprule
		Laptop & 192.168.183.158& qualität niedrig  \\
		\midrule
		i7 linux &67 &60          \\
		i7 win &53 &47           \\
		i5 Präsentation/Win &72 &60        \\
		i5 Win &88 &66        \\ 		
		\bottomrule
	\end{tabular}
\end{table}
\subsubsection{16.04}
-Einrichtung Remote Server gns3 auf testlab
- ubuntu server network 192.168.183.0/24, 192.168.183.181, gateway 192.168.183.1, dns 195.98.223.1 und dann installscripts \url{https://docs.gns3.com/1c2Iyiczy6efnv-TS_4Hc7p11gn03-ytz9ukgwFfckDk/index.html}
- web ui zugang auf 192.168.183.181:3080
- gns3-converter um remote triggered Black Holing projekt zu importieren
\url{https://blog.pierky.com/gns3-lab-remote-triggered-black-holing/}
- in vim .net File alle Verbindungen entfernt, da Ethernet port mit serial port connected war
- rm -r Downloads*
-gns3-converter ~/Downloads/RTBH.net
-vim RTBH.net
-gns3 neustarten und open project
-mit Eric Switch im RZ2 eingebaut
\subsubsection{17.04}
-projekt vom blog auf remote gns3 server mit c7200 routern nachgebaut
- blackholing global oder bestimmte isp funktioniert
-nachbau backbone area über draw.io bild und versuch blackholing damit zu simulieren

\subsection{8. Woche}
\subsubsection{20.04}
-KVM wieder eine KVM starten zu können auf HPV1 folgendes umgesetzt in Proxmox bewirkt flag vmx|svm bei /proc/cpuinfo
- GNS3 Project über webui exportieren und in neuem server wieder über webui importieren
- nicht die images mit scp hinüberkopieren, sondern auf Client mit gns3-gui das template für Router/Switch erstellen und damit hochladen
- dann projekt über web ui exportieren und dann wieder importieren
-Szenario Remote Triggerd Black Holing für Planet Back Bone Area nachgebaut bis Schritt 2.Routing der rtbh configs, also 3 AS für bgp in den routern und die ip adressen verteilt
\subsubsection{21.04}
-mit eric asa im rz eingebaut
-linphone für desktop eingerichtet sip:2171@troubadix.planet-ic.de <sip:troubadix.planet-ic.de;transport=udp> dann oben account auswählen und passwort eingeben
-backbone area planet in gns3 ohne core router nach blog-beispiel weitergebaut \url{https://blog.pierky.com/gns3-lab-remote-triggered-black-holing/}
- router konfigurationen für ISP Edge und Cust Router mit ip adressen und bgp mit filter list out für edge router
-access list und route maps nummer am ende 10/20/30 ist einfach nummerierung \url{https://www.cisco.com/c/en/us/support/docs/ip/border-gateway-protocol-bgp/49111-route-map-bestp.html} match und set bei route-maps

\subsubsection{22.04}
-versuch customer triggered black holing anzuwenden, da advertising der bgp prefix routen nicht geklappt hat \url{https://blog.pierky.com/gns3-labs-bgp-customer-triggered-black-holing/}
- gelernt über route maps und bgp verbindungen mit neighbor angaben auf beiden routerseiten(Core und Edge), haben bei customer black holing aber auch nicht funktioniert
- letztlich doch mit der remote variante funktioniert, bb05 ist core router mit den vier/fünf RTBH route map regeln, redistributed die static route map und baut die Edge peer-group auf
- immer mit sh ip bgp $prefix$ oder sh ip route/cef auf den edge routern die verbindung prüfen und dann von ISP ping 192.168.10.1 source lo0
- bei den eingehenden edge routern mit der route map auf community und as path prüfen und dann auf 192.0.2.1 die wiederum auf null0 umleitet
- BGP-CUST1 aus den beispielen kann vernachlässigt werden, nur damit customer nur ihre eigenen Netze anbieten
- es war zwingend erforderlich das ospf 300 netz für core und edges einzurichten, erst dann ging die Übergabe der bgp prefix routen, die bei Core eingegeben werden --> ip route 192.168.10.1 255.255.255.255 null0 tag 100 bisher getestet
-lag möglicherweise an set origin igp
\subsubsection{23.04}
-weitere Versuche mit customer black holing auf 192.168.183.181 projekt blackholing
- route wird von cust 10 angegeben, Edge3 darf diese nicht in einer eingehenden routemap mit set verändern, sondern nur mit match prüfen und dann weiter durchreichen. Sobald set ip next-hop 192.0.2.1 gemacht wird, wird die route nicht mehr an edge1 und edge2 und auch nicht an core geschickt.  set community no-export no-advertise geht nur ohne no-advertise und dann wird bei sh ip bgp next hop 192.168.0.2 statt 192.0.2.1 genommen.
- problem ist die weiterverteilung der routen über bgp, so dass jede node entsprechend seiner community die regel beachtet oder denied
- das remote black holing triggern über core funktioniert weiterhin bei 181 auf aufgabenstand 4 der zusammengeschusterten rtbh configs und bei 182 noch auf den ursprünglichen RTBH 3 configs
- bei 182 mit planet backbone project ist der core router code in einen edge router reingetan und das isp blackholing funktioniert, allerdings ist global connect(der core/edge router) immer gesperrt und nicht entsprechend der community(bei 102/199 sollte er durchlassen, tut er aber nicht)
- customer sind nicht über einen edge router verbunden, sondern direkt über switch deswegen wird durch tag 199 nicht das andere customer netz geblackhol'ed, sondern für den dritten isp genutzt (DTAG)
-mit eric server ausgebaut und rambestückungen getestet

\subsection{10. Woche}
\subsubsection{04.05}
-andreas gns3 einführung gegeben
- sudo gns3restore hat gns3 virtualbox vm zerschossen, wieder zum informations screen mit zweimal exit oder logout
- kopieren der projekte von remote server zu lokal
-vorbereiten von mdk test mit bgp ausfallsichere routen
\section{Introduction}\label{sec:intro}
- Tätigkeiten im Unternehmen/beim Kunden, Ansprechpartner fallen wegen Corona möglicherweise aus, andere Aufgaben vorziehen
- Nachricht von Bundesnetzagentur systemrelevanter betrieb

\lipsum[1-3]\todo{Refine me}

The remainder of the paper starts with a presentation of related work (\cref{sec:relatedwork}).
It is followed by a presentation of hints on \LaTeX{} (\cref{sec:hints}).
Finally, a conclusion is drawn and outlook on future work is made (\cref{sec:outlook}).

\section{Related Work}
\label{sec:relatedwork}

Winery~\cite{Winery} is a graphical \commentontext{modeling}{modeling with one \qq{l}, because of AE} tool.
The whole idea of TOSCA is explained by \citet{Binz2009}.

\section{LaTeX Hints}
\label{sec:hints}

\begin{figure}
  \centering
  \includegraphics[width=.8\textwidth]{example-image-golden}
  \caption{Simple Figure. \cite[based on][]{mwe}}
  \label{fig:simple}
\end{figure}

\begin{table}
  \caption{Simple Table}
  \label{tab:simple}
  \centering
  \begin{tabular}{ll}
    \toprule
    Heading1 & Heading2 \\
    \midrule
    One      & Two      \\
    Thee     & Four     \\
    \bottomrule
  \end{tabular}
\end{table}

\begin{lstlisting}[
  % one can adjust spacing here if required
  % aboveskip=2.5\baselineskip,
  % belowskip=-.8\baselineskip,
  caption={Example Java Listing},
  label=L1,
  language=Java,
  float]
public class Hello {
    public static void main (String[] args) {
        System.out.println("Hello World!");
    }
}
\end{lstlisting}

\begin{lstlisting}[
  % one can adjust spacing here if required
  % aboveskip=2.5\baselineskip,
  % belowskip=-.8\baselineskip,
  caption={Example XML Listing},
  label=L2,
  language=XML,
  float]
<example attr="demo">
  text content
</example>
\end{lstlisting}

\Cref{L1,L2} show listings typeset using the \texttt{lstlisting} environment.

cref Demonstration: Cref at beginning of sentence, cref in all other cases.

\Cref{fig:simple} shows a simple fact, although \cref{fig:simple} could also show something else.

\Cref{tab:simple} shows a simple fact, although \cref{tab:simple} could also show something else.

\Cref{sec:intro} shows a simple fact, although \cref{sec:intro} could also show something else.

Brackets work as designed:
<test>
One can also input backquotes in verbatim text: \verb|`test`|.

The symbol for powerset is now correct: $\powerset$ and not a Weierstrass p ($\wp$).

\begin{inparaenum}
  \item All these items...
  \item ...appear in one line
  \item This is enabled by the paralist package.
\end{inparaenum}

Please use the \qq{qq command} or the \enquote{enquote command} to quote something.
``something in quotes'' using plain tex syntax also works.

You can now write words containing hyphens which are hyphenated (application"=specific) at other places.
This is enabled by an additional configuration of the babel package.
In case you write \qq{application-specific}, then the word will only be hyphenated at the dash.
You can also write applica\allowbreak{}tion-specific, but this is much more effort.

The words \qq{workflow} and \qq{dwarflike} can be copied from the PDF and pasted to a text file.

Numbers can written plain text (such as 100), by using the siunitx package like that:
\SI{100}{\km\per\hour},
or by using plain \LaTeX{} (and math mode):
$100 \frac{\mathit{km}}{h}$.

\section{Conclusion and Outlook}
\label{sec:outlook}
\lipsum[1-2]

\subsubsection*{Acknowledgments}
\ldots

In the bibliography, use \texttt{\textbackslash textsuperscript} for \qq{st}, \qq{nd}, \ldots:
E.g., \qq{The 2\textsuperscript{nd} conference on examples}.
When you use \href{https://www.jabref.org}{JabRef}, you can use the clean up command to achieve that.
See \url{https://help.jabref.org/en/CleanupEntries} for an overview of the cleanup functionality.

\renewcommand{\bibsection}{\section*{References}} % requried for natbib to have "References" printed and as section*, not chapter*
% Use natbib compatbile splncsnat style.
% It does provide all features of splncs03, but is developed in a clean way.
% Source: http://phaseportrait.blogspot.de/2011/02/natbib-compatible-bibtex-style-bst-file.html
\bibliographystyle{splncsnat}
\begingroup
  \ifluatex
    %try to activate if bibliography looks ugly
    %\sloppy
  \else
    \microtypecontext{expansion=sloppy}
  \fi
  \small % ensure correct font size for the bibliography
  \bibliography{paper}
\endgroup

% Enfore empty line after bibliography
\ \\
%
All links were last followed on October 5, 2017.
\end{document}
